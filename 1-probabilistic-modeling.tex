\section{Probabilistic Modeling}
\subsection{Probability Theory}
\marginnote{In a single coin toss, we have \[\Omega = \{\text{H}, \text{T}\}\]}
We will briefly review probability in a rigorous sense. \\We define events considering we have a space of possible outcomes denoted by $\Omega$. $\mathcal S$ is a set of measurable events, to which we assign probabilities, and each event $\alpha \in \mathcal S$ is a subset of $\Omega$. \\
The event space necessarily satisfies three properties - 
\begin{enumerate}
	\item It contains the empty event $\emptyset$ and the trivial event $\Omega$.
	\item It is closed under union, i.e if $\alpha, \beta \in \mathcal S$, so is $\alpha \cup \beta$.
	\item It is closed under complementation, i.e if $\alpha \in \mathcal S$, so is $\Omega - \alpha$. 
\end{enumerate}
\begin{defn}[Probability distribution]\label{defn:prob}
A probability distribution $P$ over $(\Omega, \mathcal S)$ is a mapping of events in $\mathcal S$ to real values satisfying 
\marginnote{A direct consequence of the properties is: \[P(\emptyset) = 0\]\[P(\alpha \cup \beta) = P(\alpha) + P(\beta) - P(\alpha \cap \beta)\]}
\begin{itemize}[leftmargin=1cm]
	\item[$\diamond$] $P(\alpha) \geq 0$ for all $\alpha \in \mathcal S$
	\item[$\diamond$] $P(\Omega) = 1$
	\item[$\diamond$] If $\alpha, \beta \in \mathcal S$ and $\alpha \cap \beta = \emptyset$, then $P(\alpha \cup \beta) = P(\alpha) + P(\beta)$
\end{itemize}
\end{defn}
Conditional probability answers the question - after learning that event $\alpha$ is true, how does our belief about $\beta$ change? Formally, we define
\begin{equation}
	P(\beta|\alpha) = \dfrac{P(\alpha \cap \beta)}{P(\alpha)}
\end{equation}
It can be checked that this satisfies Definition \ref{defn:prob} and is a probability distribution. Noting that $P(\alpha \cap \beta) = P(\alpha) P(\beta|\alpha)$, we define the chain rule of conditional probabilities
\begin{equation}
P(\alpha_1 \cap \cdots\cap \alpha_k) = P(\alpha_1) P(\alpha_2|\alpha_1) \cdots P(\alpha_k|\alpha_1\cap\cdots\cap\alpha_{k-1})
\end{equation}
We further define the Bayes' rule
\begin{equation}\label{eqn:bayes}
P(\alpha|\beta) = \dfrac{P(\beta|\alpha)P(\alpha)}{P(\beta)}
\end{equation}
Here, $P(\alpha|\beta)$ is called the \textit{posterior}, $P(\beta|\alpha)$ is the \textit{likelihood}, $P(\alpha)$ is the \textit{prior} and $P(\beta)$ is the \textit{marginal probability} of the structure in context. We can generalize Equation \ref{eqn:bayes} as
\begin{equation}
P(\alpha|\beta \cap \gamma) = \dfrac{P(\beta|\alpha \cap \gamma)P(\alpha|\gamma)}{P(\beta|\gamma)}
\end{equation}
Now, we formally define the notion of random variables, which intuitively can be considered to be attribute reporters.
\begin{defn}[Random Variable]
	A random variable $X$ is a \textit{measurable} function $X:\Omega \to \mathcal S$. The probability that $X$ takes values in a set $s \in \mathcal S$ is written as
	\begin{equation}
		\text{Pr}(X \in s) = \text{Pr}(\{\omega \in \Omega|X(\omega) \in s\})
	\end{equation}
\end{defn}
The marginal distribution over a random variable $X$ is the distribution over events that can be described using $X$, and is denoted by $P(X)$. More generally, if we want to describe a distribution over a set of random variables $\mathcal X = \{x_1, \cdots, x_n\}$ called the \textit{joint distribution} denoted as $P(x_1, \cdots, x_n)$. The full assignment to the variables is denoted as $\xi \in \text{Val}(\mathcal X)$. The space corresponding to the joint assignment in $\mathcal X$ is called the \textit{canonical outcome space}. \\
Now, we glance at independencies, a core component of Probabilistic Graphical Models.
\begin{defn}[Independence]
An event $\alpha$ is independent of an event $\beta$ denoted by $P \models (\alpha\ind\beta)$, if $P(\alpha|\beta) = P(\alpha)$ or $P(\beta) = 0$.
\end{defn}
\begin{prop}
A distribution satisfies $(\alpha \ind \beta)$ if and only if 
\begin{equation}
P(\alpha\cap\beta) = P(\alpha)P(\beta)
\end{equation}
\end{prop}
\begin{proof}
Skipped (hint: Use the definition of conditional probability).
\end{proof}
\begin{defn}[Conditional Indpendence]
An event $\alpha$ is conditionally independent of event $\beta$ given $\gamma$ in $P$, denoted by $P \models (\alpha \ind\beta|\gamma)$ if $P(\alpha|\beta\cap\gamma) = P(\alpha|\gamma)$ or if $P(\beta\cap\gamma) = 0$.
\end{defn}
\begin{prop}
$P$ satisfies $(\alpha\ind\beta|\gamma)$ if and only if 
\begin{equation}
P(\alpha \cap\beta|\gamma) = P(\alpha|\gamma)P(\beta|\gamma)
\end{equation}
\end{prop}
\begin{defn}
Let $\mathbf{X, Y, Z}$ be sets of random variables. $\mathbf X$ is conditionally independent of $\mathbf Y$ given $\mathbf Z$ in a distribution $P$ if $P$ satisfies $(\mathbf X = \mathbf x \ind \mathbf Y = \mathbf y | \mathbf Z = \mathbf z)$ for all values of $\mathbf x \in \text{Val}(\mathbf X), \mathbf y \in \text{Val}(\mathbf Y)$ and $\mathbf z \in \text{Val}(\mathbf Z)$. We say that the variables in $\mathbf Z$ are \textit{observed}. If $\mathbf Z$ is empty, then we say that $\mathbf X$ and $\mathbf Y$ are marginally independent.
\end{defn}
\begin{prop}
The distribution $P$ satisfies $(\mathbf X \ind \mathbf Y|\mathbf Z)$ if and only if 
\begin{equation}
	P(\mathbf X,\mathbf Y|\mathbf Z) = P(\mathbf X|\mathbf Z)P(\mathbf Y|\mathbf Z)
\end{equation}
\end{prop}
The following properties hold for conditional independencies:
\begin{enumerate}
	\item \textit{Symmetry:} $(\mathbf X \ind \mathbf Y|\mathbf Z) \implies (\mathbf Y \ind \mathbf X|\mathbf Z)$
	\item \textit{Decomposition:} $(\mathbf X \ind \mathbf{Y, W}|\mathbf Z) \implies (\mathbf X \ind \mathbf Y|\mathbf{Z})$
	\item \textit{Weak Union:} $(\mathbf X \ind \mathbf Y, \mathbf W|\mathbf Z) \implies (\mathbf X \ind \mathbf Y|\mathbf Z,  \mathbf W)$
	\item \textit{Contraction:} $(\mathbf X \ind \mathbf W|\mathbf Z, \mathbf Y)\; \& \;(\mathbf X \ind \mathbf Y|\mathbf Z) \implies (\mathbf X \ind \mathbf Y, \mathbf W|\mathbf Z)$
\end{enumerate}
If our distribution is positive (i.e, for all non-empty $\alpha \in \mathcal S, P(\alpha) > 0$), we have another property
\begin{itemize}[leftmargin=1cm]
	\item[$\diamond$] \textit{Intersection:}  $(\mathbf X \ind \mathbf Y|\mathbf Z, \mathbf Z)\; \& \;(\mathbf X \ind \mathbf W|\mathbf Z, \mathbf Y) \implies (\mathbf X \ind \mathbf Y, \mathbf W|\mathbf Z)$
\end{itemize}

\subsection{Probabilistic Graphical Models}
Given a set of $n$ random variables $\mathcal{X} = \{x_1, x_2, \cdots x_n\}$ where $n$ is large, we want to build a joint probability distribution $P$ over this set. Explicitly representing the joint distribution is computationally expensive, since just having binary values variables requires the joint distribution to specify $2^n-1$ numbers, and for more practical variables, the count is too large. \\
\marginnote{An example of a query can be - \texttt{Estimate the fraction of people with a bachelor's degree}.}
We want to efficiently represent, estimate and answer inference queries on the distribution.
\subsection{Alternatives to explicit joint distributions}
$\triangleright$ Can we assume all columns are independent? \textbf{NO} - this is obviously a very bad assumption. \\
\noindent$\triangleright$ Can we use data to detect highly correlated column pairs, and estimate their pairwise frequencies? \textbf{MAYBE} - but there might\\\noindent be too many correlated pairs, and the method is ad hoc.\\
\marginnote{Note that we write that a set $X$ is conditionally independent of $Y$ given $Z$, i.e $X \ind Y|Z$ if \[ \Prob(X|Y,Z) = \Prob(X|Z)\]}
\noindent To solve the above two not so good ways, we explore conditional independencies. It may be possible that $\texttt{income } \cancel{\ind} \texttt{ age}$ but \\\noindent$\texttt{income } {\ind} \texttt{ age}|\texttt{experience}$. \\
Probabilistic graphical models use a graph-based representation as the basis for compactly
encoding a complex distribution over a high-dimensional space. \\
It is convenient to represent the independence assumption using a graph. The so called graphical model has nodes as the variables (continuous or discrete), and the edges represent direct interaction. If we consider directed edges, we talk about Bayesian Networks, and if we consider undirected edges, we talk about Markov Random Fields. \\
Essentially the graphical model is a combination of the graph and potentials.
\begin{defn}[Potentials]
Potentials $\psi_c(\mathbf x_c)$ are scores for assignment of values to subsets c of directly
interacting variables. We factorize the probability as a product of these potentials, i.e
\begin{equation}
	\text{Pr}(\mathbf x = x_1, \cdots, x_n) \propto \prod \psi_s(\mathbf x_s)
\end{equation}
\end{defn}

