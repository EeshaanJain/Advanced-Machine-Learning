\section{Bayesian Networks}
Bayesian Networks, also referred to as \textit{directed graphical models} are a family of probability distributions that has a compact parameterization representable using a directed graph. \\
It is known that
\begin{equation}
\Prob(x_1, x_2, \cdots, x_n) = \Prob(x_1)\Prob(x_2|x_1)\Prob(x_3|x_2,x_1)\cdots\Prob(x_n|x_{n-1}, \cdots, x_1)
\end{equation}
A compact Bayesian Network is a distribution in which each factor in the above equation depends on the \textit{parent} variables represented by Pa($x_i$) for variable $x_i$. Thus, we have
\begin{equation}
\Prob(x_i|x_{i-1}, x_{i-2}, \cdots ,x_1) = \Prob(x_i|\text{Pa}(x_i))
\end{equation}
and the corresponding potentials at each node in terms of its parents are
\begin{equation}
\psi_i(x_i, \text{Pa}(x_i)) = \Prob(x_i, \text{Pa}(x_i))
\end{equation}
Thus,
\begin{equation}\label{eqn:factorization}
	\Prob(x_1, x_2, \cdots, x_n) = \prod_{i=1}^n \Prob(x_i|\text{Pa}(x_i))
\end{equation}
\marginnote{\begin{exmp}\label{exmp:bn-ind}
\end{exmp}}
\begin{marginfigure}
	\centering
	\begin{tikzpicture}[main/.style = {draw, circle}] 
		\node[main] (a) {A}; 
		\node[main] (b) [right of=a] {B};
		\node[main] (c) [below of=a] {C};
		\node[main] (d) [below of=b] {D};
		\node[main] (e) [below = of $(c)!0.5!(d)$] {E};
		\draw[->] (b) -- (d);
		\draw[->] (c) -- (e);
		\draw[->] (d) -- (e);
	\end{tikzpicture}
	\caption{Sample BN}
\end{marginfigure}
\marginnote{Consider the BN above. We will consider each variable at a time.
	\begin{itemize}
		\item[$\diamond$] $A$ has no parent, and has no descendent. Thus, \[ A \ind B, C, D, E\]
		\item[$\diamond$] $B$ has no parent, but has $D$ as a descendent. Thus, \[B \ind A, C\]
		\item[$\diamond$] $C$ has no parent, but has $E$ as a descendent. Thus, \[C \ind A, B, D\]
		\item[$\diamond$] $D$ has $B$ as a parent, and has $E$ as the descendent. Thus, \[D \ind A, C|B\]
		\item[$\diamond$] $E$ has $C$ and $D$ as parents, but has no descenent. Thus, \[I \ind A, B|C,D\]
	\end{itemize}
}
Consider the situation when each variable can take $d$ values. The naive approach gives us $\mathcal{O}(d^n)$ parameters. If we think of the potentials as probability tables (with the rows corresponding to Pa($x_i$)) and columns corresponding to the values of $x_i$, with entries as $\psi_{i}(x_i, \text{Pa}(x_i))$, we can notice that if $|\text{Pa}(x_i)| \leq k$, then the number of parameters are $\mathcal{O}(d^{k+1})$, and for $n$ variables, we have $\mathcal{O}(nd^{k+1})$, which provides us the compact representation.
\subsection{Definition}
Now we formally define these -
\begin{defn}[Bayesian Network]
A Bayesian Network is a directed graph $G = (V, E)$ together with 
\begin{itemize}
	\item[$\diamond$] a random variable $x_i$ for each node $i \in V$
	\item[$\diamond$] a potential $\psi_i(x_i, \text{Pa}(x_i))$ for each node $i \in V$
\end{itemize}
\end{defn}
For a variable $x_i$ in our Bayesian Network $\mathcal G$, denote $\text{ND}(x_i)$ as the non-descendents of $x_i$. The following local conditional independencies hold in $\mathcal G$ - 
\begin{equation}
	x_i \ind \text{ND}(x_i) | \text{Pa}(x_i)
\end{equation}
Example \ref{exmp:bn-ind} shows the independencies in a simple Bayesian Network.
\begin{defn}[Factorization]	
Let $\mathcal{G}$ be a Bayesian Network graph over the variables $\{X_i\}_{i=1}^n$. We say that a distribution $P$ over the same space factorizes according to $\mathcal G$ if $P$ can be expressed as a product described in Equation \ref{eqn:factorization}. Such factorization is also known as the chain rule for Bayesian Networks, and is denoted as $\text{Factorize}(P, \mathcal G)$.
\end{defn}

\begin{defn}
Let $P$ be a distribution over $\mathcal X$. We define $\mathcal{I}(P)$ to be the set of independent assertions of the form $\mathbf X \ind \mathbf Y | \mathbf Z$ that hold in $P$.
\end{defn}
We can now write "$P$ satisfies the local independencies associated with $\mathcal  G$" as $\mathcal{I}_\ell(\mathcal G) \subseteq \mathcal{I}(P)$.
\begin{defn}[Independency-Map]
Let $\mathcal K$ be any graph object associated with a set of independencies $\mathcal{I}(\mathcal K)$. We call $\mathcal K$ an I-map for a set of independencies $\mathcal I$ if $\mathcal I(\mathcal K) \subseteq \mathcal I$.
\end{defn}
Thus for $\mathcal G$ to be an I-map for $P$, any independence that asserts in $\mathcal G$ must also assert in $P$, but $P$ can have additional independencies not reflected in $\mathcal G$.
\begin{rem}[Notation Alert]
Note that we will use the following interchangeably - $P$ satisfies the local conditional independencies satisfied by $\mathcal G$ and $\mathcal G$ is an I-map for $P$, i.e
\begin{equation}
	\text{Local-CI}(P, \mathcal G) \equiv \mathcal{I(G)} \subseteq \mathcal{I}(P)
\end{equation}
\end{rem}
\begin{thm}
Given a distribution $P(x_1, x_2, \cdots, x_n)$ and a directed acyclic graph (DAG) $\mathcal{G}$, 
\begin{equation}
\text{Local-CI}(P, \mathcal G) \Longleftrightarrow \text{Factorize}(P, \mathcal G)
\end{equation}
\end{thm}
\begin{proof}
($\implies$) We essentially need to show that if $\mathcal{G}$ is an I-map for $P$, then $P$ factorizes according to $\mathcal G$. Consider a topologically sorted order $x_1, x_2, \cdots, x_n$ in $\mathcal G$. $\text{Local-CI}(P, \mathcal G)$ tells us that $$\text{Pr}(x_i|x_{1}, \cdots, x_{i-1}) = \text{Pr}(x_i|\text{Pa}(x_i))$$
We can write
$$P(x_1, x_2, \cdots, x_n) = \prod_{i=1}^nP(x_i|x_1, \cdots, x_{i-1})$$
Each term in the product can be simplified due to the notion of Local-CI stated above, and we reach Equation \ref{eqn:factorization}, proving factorization. \\
($\impliedby$) Proof has been skipped.
\end{proof}
\subsection{Minimal Construction}
Our goal is to construct a minimal and correct BN $\mathcal G$ to represent $P$. A DAG $\mathcal G$ is correct if all Local-CIs that are implied in $\mathcal G$ hold in $P$, and a DAG $\mathcal G$ is minimal if we cannot remove any edge(s) from $\mathcal G$ and still get a correct BN for $P$.

 In the setting, we define our oracle $\mathscr{O}$ to whom we can ask any query of the type "\texttt{Is X $\ind$ Y|Z}?" pertaining to $P$ and get a boolean answer. We will query the oracle several times to build up our BN. The following algorithm constructs such a BN - \\
\begin{algorithm}[H]\label{alg:bn-con}
	\DontPrintSemicolon
	\textbf{Variables:} $x_1, x_2, \cdots, x_n \longleftarrow$ ordered variables in $\mathcal{X}$\;
	\textbf{Independencies:} $\mathcal I \longleftarrow$ set of independencies\;
	$\mathcal G \longleftarrow$ Empty graph over $\mathcal X$\;
	\For{$i=1$ to $n$}{
		$\mathbf U \longleftarrow \{x_1, \cdots, x_{i-1}\}$ \Comment{Set of candidate parents of $x_i$} \;
		\For{$U' \subseteq\{x_1, \cdots, x_{i-1}\}$}{
		\If{$U' \subset U$ and $(x_i \ind \{x_1, \cdots, x_{i-1}\} - U'|U')\in\mathcal I $}{
			$\mathbf{U} \longleftarrow U'$\;
			}	
	}
	\Comment{Now we have the minimal set $\mathbf{U}$ satisfying $(x_i \ind \{x_1, \cdots, x_{i-1}\} - \mathbf U|\mathbf U)$} \;
	\Comment{Now we set $\mathbf{U}$ to be the parents of $x_i$}\;
	\For{$x_j \in \mathbf U$}{
			Add $x_j \to x_i$ in $\mathcal G$
		}
	}
	\Return{$\mathcal{G}$}
	\caption{Minimal Bayesian Network Construction (I-Map)}
\end{algorithm}
We know sketch rough proofs for the claims of the algorithm.
\begin{thm}
The BN $\mathcal G$ constructed by algorithm \ref{alg:bn-con} is minimal, i.e we cannot remove any edge from the BN while maintaining the correctness of the BN for $P$.
\end{thm}
\begin{proof}
By construction. A subset of $\text{ND}(x_i)$ were available when we chose parents of $\mathbf U$ minimally.
\end{proof}
\begin{thm}
$\mathcal{G}$ constructed by the above algorithm is correct, i.e, the local-CIs induced by $\mathcal G$ hold in $P$.
\end{thm}
\begin{proof}
The construction is such that $\text{Factorize}(P, \mathcal G)$ holds everytime. Since $\text{Factorize}(P, \mathcal G) \implies \text{Local-CI}(P, \mathcal G)$, the constructed BN satisfies the local-CIs of $P$.
\end{proof}
\marginnote{\begin{exmp}\label{exmp:bn-order}
To be added.
\end{exmp}}
\begin{rem}[Importance of ordering]
It is possible that a different ordering in $\mathcal X$ gives rise to a different BN, which although may be minimal, but may not be \textit{optimal}. A minimal BN is defined for a given ordering, while an optimal BN is defined over all orderings. Example \ref{exmp:bn-order} shows such a case.
\end{rem}
\subsection{D-Separation}
Our goal is to know when we can guarantee $\mathbf X \ind \mathbf Y|\mathbf Z$ holds given a BN  $\mathcal G$. The further discussion provides some cases where we can guarantee $\mathbf X\; \cancel\ind\; \mathbf Y|\mathbf Z$.
\begin{enumerate}
	\item \textbf{Direct Connection:} If there is an edge $X \to Y$, then regardless of any $\mathbf Z$, we can find examples where they influence each other.
	\item \textbf{Indirect Connection:} [To be added]
\end{enumerate}